\documentclass{article}
\usepackage{amsmath}
\usepackage{amssymb}
\usepackage{url}
\usepackage{bm}
\usepackage{float}
\usepackage{comment}
\usepackage{graphicx}
\graphicspath{ {Image/} }
\usepackage{geometry}
%\geometry{left=2.cm, right=2.cm, top=2.54cm, bottom=2.54cm}
\usepackage[final]{hyperref} 
\hypersetup{
    colorlinks=true,
    linkcolor=blue,
    citecolor=blue,  
    filecolor=magenta, 
    urlcolor=blue         
}

\author{Chenxiao Zeng}
\date{\today}
\title{\textbf{CCAPP Research Summary: Modeling of EoR Ionization Front}\\\vspace{20pt}Fall 2018}
\begin{document}	
\maketitle
\tableofcontents

\section{Annotated Bibliography}

\section{Introduction}
Please refer to the introduction section in \cite{Hirata:2018ss}.
\section{Analytic Analysis}
We are considering interaction between five species: electron ($e^-$), hydrogen ion ($\text{H}^+$), helium ion ($\text{He}^+$), neutral hydrogen ($\text{H}$), and neutral helium ($\text{He}$).
Some properties are listed in Table.\ref{species_properties}.

\begin{table}[ht]
\begin{center}
\caption{Useful properties of species.}
\label{species_properties}
\begin{tabular}{cccccc} \hline \hline
\multicolumn{1}{c}{Parameters}&  \multicolumn{1}{c}{$e^-$} & \multicolumn{1}{c}{$\text{H}^+$} & \multicolumn{1}{c}{$\text{He}^+$} & \multicolumn{1}{c}{$\text{H}$} & \multicolumn{1}{c}{$\text{He}$}\\
\hline
Mass (kg) & $9.109 \times 10^{-31}$ & $1.673 \times 10^{-27}$ & $6.647 \times 10^{-27}$  & $1.673 \times 10^{-27}$ & $6.647 \times 10^{-27}$ \\
Mass (MeV/$c^2$) & 0.511 & 938.272 & 3727.911 & 938.896 & 3728.422\\
$\ln \Lambda$ &  &  &  &  & \\
\hline \hline
\end{tabular}
\end{center}
\end{table}

In the process we are discussing, the interactions are sorted in five groups: electrons and ions, ions and ions, electrons and neutrals, ions and neutrals, and between neutrals and neutrals. We aim to generate a $5\time5$ matrix of energy transferring rate between species. For all species we adopt Maxwellian distributions $f(v)$ for calculations of speeds.

\subsection{Basic Principles of Plasma Physics}
The average kinetic energy of a particle in three dimensions is
\begin{equation}
	E_{av} = \frac{3}{2}kT = 3 \cdot \frac{1}{2} m v^2
\end{equation}
Accordingly, its average thermal speed is
\begin{equation}
	v = \Big(\frac{kT}{m} \Big)^{1/2}
\end{equation}
In addition, quasi-neutrality demands that
\begin{equation}
	n_i (\text{total ion species}) \simeq n_e \equiv n
\end{equation}
The strategy I adopt is illustrated as following:
\begin{equation*}
	\begin{split}
		&\text{relative velocity distribution} \times \text{energy loss rate for single collision} \rightarrow \text{distributional energy loss }
	\end{split}
\end{equation*}
\subsection{Electron $\Longleftrightarrow$ Ions}
In this section we aim to find the momentum and energy loss rate when electrons collide ions. To do so, we need two components: the thermal distribution of electrons and the collision rate as a function of velocity. 

In the stationary reference frame of ions, the electrons obey the velocity distributions of \cite{Fitzpatrick:2014pp}:
\begin{equation}
	\label{eq_f_e}
	f_e(\bm{v}) = n_e \Big(\frac{m_e}{2\pi kT_e} \Big)^{3/2} \exp\Big[-\frac{m_e(\bm{v} - \bm{v_d})^2}{2kT_e} \Big]
\end{equation}
where $\bm{v_d}$ is the drift velocity of the electrons ensemble, and $\bm{v}$ is the velocity of electrons \underline{\textbf{relative}} to that of ions. We do not include the thermal motion of the ions because they are moving slowly compared with electrons.

Consider a single electron collides with an ion, the momentum collision frequency $\nu_{ei,p}$ ($p$ denotes momentum) is \cite{Chen:2016it}:
\begin{equation}
	\label{eq_nu_ei}
	\nu_{ei,p} = n_i\frac{q_e^2 q_i^2}{(4\pi \epsilon_0 )^2}\frac{4\pi(m_e+m_i)}{m_i m_e^2 v^3} \ln \Lambda_e
\end{equation}
where $\ln \Lambda$ is the Coulomb Logarithm , and according to \cite{Draine:2011po},
\begin{equation}
	\label{eq_lambda_e}
	\ln \Lambda_e = 22.1 + \ln \Big[\Big(\frac{E_e}{kT_e} \Big)\Big(\frac{T_e}{10^4 \text{K}} \Big)\Big(\frac{\text{cm}^{-3}}{n_e} \Big) \Big]
\end{equation}
By introducing the Coulomb Logarithm factor, we constrain the lower and upper cutoffs when integrating over all relevant impact parameters $b$.

Therefore, combining Eq.\ref{eq_f_e}, \ref{eq_nu_ei}, and \ref{eq_lambda_e}, we obtain the total momentum loss rate per unit volume, assuming that $\ln \Lambda$ is a constant instead of function of $v$, and that $|\bm{v}_d| \ll v$:
\begin{equation}
	\begin{split}
		-\frac{d \bm{p}_e}{d t} &= \int d^3v\ f_e(\bm{v}) \nu_{ei,p} (v) m_e \bm{v}\\
		&= \int d^3v\ m_e\bm{v} n_e \Big(\frac{m_e}{2\pi kT_e} \Big)^{3/2} \exp\Big[-\frac{m_e(\bm{v} - \bm{v_d})^2}{2kT_e} \Big] \cdot n_i\frac{q_e^2 q_i^2}{(4\pi \epsilon_0 )^2}\frac{4\pi(m_e+m_i)}{m_i m_e^2 v^3} \ln \Lambda_e \\
		&= m_e n_e \Big(\frac{1}{2\pi v_e^2} \Big)^{3/2} \cdot n_i\frac{q_e^2 q_i^2}{(4\pi \epsilon_0 )^2}\frac{4\pi(m_e+m_i)}{m_i m_e^2} \ln \Lambda_e \int d^3v\ \frac{\bm{v}}{v^3} \exp\Big[-\frac{m_e(\bm{v} - \bm{v_d})^2}{2kT_e} \Big] 
	\end{split}
\end{equation}
where $v_e = (kT_e/m )^{1/2}$ is the average thermal speed at temperature $T_e$. In addition, we can define the the integral as
\begin{equation}
	\begin{split}
		I &\equiv \int d^3v\ \frac{\bm{v}}{v^3} \exp\Big[-\frac{m_e(\bm{v} - \bm{v}_d)^2}{2kT_e} \Big] \\
		&= \int d^3v\ \frac{\bm{v}}{v^3} \exp\Big[-\frac{m_e}{2kT_e}(\bm{v} - \bm{v}_d)^2 \Big]\\
		&= \int d^3v\ \frac{\bm{v}}{v^3} \exp\Big[-\frac{m_e}{2kT_e}(\bm{v}\cdot\bm{v} - 2\bm{v}\cdot\bm{v}_d + \bm{v}_d\cdot\bm{v}_d) \Big] \\
		&= \int d^3v\ \frac{\bm{v}}{v^3} \exp\Big[-\frac{m_e}{2kT_e}(\bm{v}\cdot\bm{v}) \Big] \exp\Big[\frac{m_e}{T_e}(\bm{v}\cdot\bm{v}_d ) \Big] \exp\Big[-\frac{m_e}{2kT_e}(\bm{v}_d\cdot\bm{v}_d) \Big] \\
		&= \int d^3v\ \frac{\bm{v}}{v^3} \exp\Big[-\frac{m_e}{2m_ev_e^2}(\bm{v}\cdot\bm{v}) \Big] \exp\Big[\frac{m_e}{m_e v_e^2}(\bm{v}\cdot\bm{v}_d ) \Big] \exp\Big[-\frac{m_e}{2m_ev_e^2}(\bm{v}_d\cdot\bm{v}_d) \Big] \\
		&= \int d^3v\ \frac{\bm{v}}{v^3} \exp\Big[-\frac{1}{2v_e^2}(\bm{v}\cdot\bm{v}) \Big] \exp\Big[\frac{1}{v_e^2}(\bm{v}\cdot\bm{v}_d ) \Big] \exp\Big[-\frac{1}{2v_e^2}(\bm{v}_d\cdot\bm{v}_d) \Big]\\
		&= \exp\Big[-\frac{v_d^2}{2v_e^2} \Big] \int d^3v\ \frac{\bm{v}}{v^3} \exp\Big[-\frac{v^2}{2v_e^2} \Big] \exp\Big[\frac{1}{v_e^2}(\bm{v}\cdot\bm{v}_d ) \Big]
 	\end{split}
\end{equation}
For one component of the momentum vector $\bm{p}_e$, e.g. $p_{ex}$, we can align the corresponding axis to the drift velocity $\bm{v}_d$. Therefore, the integral $I$ becomes
\begin{equation}
	\begin{split}
		I &=\exp\Big[-\frac{v_d^2}{2v_e^2} \Big] \int d^3v\ \frac{v_x}{v^3} \exp\Big[-\frac{v^2}{2v_e^2} \Big] \exp\Big[\frac{1}{v_e^2}(v_x v_d ) \Big]
	\end{split}
\end{equation}
Assuming the Maxwellian distribution is isotropic, we can establish the relation such that $3v_x^2 = v^2$, or $v_x = v/ \sqrt{3}$. Therefore,
\begin{equation}
	\begin{split}
		I &= \exp\Big[-\frac{v_d^2}{2v_e^2} \Big] \int_0^\infty d^3v\ \frac{v}{\sqrt{3} v^3} \exp\Big[-\frac{1}{2v_e^2}v^2 \Big] \exp\Big[\frac{v_d}{\sqrt{3} v_e^2}v \Big] \\
		&= \exp\Big[-\frac{v_d^2}{2v_e^2} \Big] \int_0^\infty d^3v\ \frac{1}{\sqrt{3} v^2} \exp\Big[-\frac{1}{2v_e^2}v^2 \Big] \exp\Big[\frac{v_d}{\sqrt{3} v_e^2}v \Big] \\
		&= \exp\Big[-\frac{v_d^2}{2v_e^2} \Big] \int_0^\infty dv\ 4\pi v^2 \frac{1}{\sqrt{3} v^2} \exp\Big[-\frac{1}{2v_e^2}v^2 \Big] \exp\Big[\frac{v_d}{\sqrt{3} v_e^2}v \Big] \\
		&= \frac{4\pi}{\sqrt{3}} \exp\Big[-\frac{v_d^2}{2v_e^2} \Big] \int_0^\infty dv\ \exp\Big[-\frac{1}{2v_e^2}v^2 \Big] \exp\Big[\frac{v_d}{\sqrt{3} v_e^2}v \Big] \\
		&= \frac{4\pi}{\sqrt{3}} \exp\Big[-\frac{v_d^2}{2v_e^2} \Big] \exp\Big[\frac{v_d^2}{6v_e^2} \Big]\sqrt{\frac{\pi}{2}}v_e \Big(\text{Erf}(\frac{v_d}{\sqrt{6}v_e} ) + \lim_{a\to \infty} \text{Erf}(\frac{3a - \sqrt{3}v_d}{3\sqrt{2}v_e} )  \Big) \\
		&= \frac{4\pi}{\sqrt{3}} \exp\Big[-\frac{v_d^2}{2v_e^2} \Big] \exp\Big[\frac{v_d^2}{6v_e^2} \Big]\sqrt{\frac{\pi}{2}}v_e \Big(\text{Erf}(\frac{v_d}{\sqrt{6}v_e} ) + 1 \Big)
	\end{split}
\end{equation}
1. look at the papers I downloaded by Dosledge 2. try to directly calculate dE/dt, assuming two maxwellian distributions


\subsection{Ions $\Longleftrightarrow$ Ions}

\subsection{Electron $\Longleftrightarrow$ Neutrals}

\subsection{Ions $\Longleftrightarrow$ Neutrals}

\subsection{Neutrals $\Longleftrightarrow$ Neutrals}

\section{Numerical Calculation}

\label{section_model_comparision}


%%%%% Bibliography begins here %%%%%
\begin{thebibliography}{95}
\expandafter\ifx\csname natexlab\endcsname\relax\def\natexlab#1{#1}\fi
\expandafter\ifx\csname bibnamefont\endcsname\relax
  \def\bibnamefont#1{#1}\fi
\expandafter\ifx\csname bibfnamefont\endcsname\relax
  \def\bibfnamefont#1{#1}\fi
\expandafter\ifx\csname citenamefont\endcsname\relax
  \def\citenamefont#1{#1}\fi
\expandafter\ifx\csname url\endcsname\relax
  \def\url#1{\texttt{#1}}\fi
\expandafter\ifx\csname urlprefix\endcsname\relax\def\urlprefix{URL }\fi
\providecommand{\bibinfo}[2]{#2}
\providecommand{\eprint}[2][]{\url{#2}}
  
  
  %%%%%%%%%%%%% turning points model %%%%%%%%%%%%%%%
\bibitem{Hirata:2018ss}
	C.~Hirata,
	%Small-scale structure and the Lyman-α forest baryon acoustic oscillation feature
	MNRAS {\bf 474}, 2173 (2018)

\bibitem{Fitzpatrick:2014pp}
	R.~Fitzpatrick,
	%Plasma Physics: An Introduction
	\textit{Plasma Physics: An Introduction},
	CRC Press (2014)

\bibitem{Chen:2016it}
	F.~Chen,
	%Introduction to Plasma Physics and Controlled Fusion
	\textit{Introduction to Plasma Physics and Controlled Fusion},
	Springer International Publishing (2016)

\bibitem{Draine:2011po}
	B.~Draine,
	%Physics of the Interstellar and Intergalactic Medium
	\textit{Physics of the Interstellar and Intergalactic Medium},
	Princeton University Press (2011)

\bibitem{McQuinn:2016te}
	M.~McQuinn,
	%The Evolution of the Intergalactic Medium
	Annu.\ Rev.\ Astron.\ Astrophys. {\bf 54}, 313 (2016)
 

\end{thebibliography}

\end{document}